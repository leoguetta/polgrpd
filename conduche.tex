\section{Basis lifting theorem}\label{sec:basislift}

\subsection{Conduché functors}\label{subsec:conduche}
\begin{paragr}
  Let $k\geq 0$ and $C$, $D$ be \ook k-categories. Recall that a morphism $f:C\to D$ is a {\em discrete Conduché functor} if it satisfies the following conditions:
  \begin{enumerate}
  \item For each $y\in D_n$ and each $x\in C_m$ such that $f(x)=\unit m(x)$, there is a unique $z\in C_n$ such that $x=\unit m(x)$.
    \item For each pair $y_1$, $y_2$ of $i$-composable cells in $D_n$ and each $x\in C_n$ such that $f(x)=y_1\comp i y_2$, there is a unique pair $\pair{x_1}{x_2}$ of $i$-composable cells in $C_n$ such that $x=x_1\comp i x_2$, $f(x_1)=y_1$ and $f(x_2)=y_2$.
    \end{enumerate}
    In the case where $C$, $D$ are plain \oo-categories, that is $k=\omega$, we know from~\cite[Theorem 6.11]{guetta:poldcf} that if $f:C\to D$ is a discrete Conduché functor and $D$ is freely generated by a polygraph, then so is $C$.
  \end{paragr}
  \begin{remark}
     A key feature of a free \oo-category is that its generators are
     exactly the indecomposable cells, hence uniquely determined
     (see~\cite[Section 4, Proposition 8.3]{makkai:worcom}). This is
     no longer true for $k<\omega$, which makes the general case a
     priori significantly more complicated. However, we shall see that
     discrete Conduché functors still lift free \ook k-categories for any value of $k$.
  \end{remark}
    




\subsection{Lifting of free $(\omega,k)$-categories}\label{subsec:lifting}
\begin{paragr}
  xxxxxxxx
\end{paragr}
\begin{theorem}
  xxxxxxxx
\end{theorem}
