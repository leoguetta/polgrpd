\section{Basis lifting theorem}\label{sec:basislift}

\subsection{Conduché fibrations}\label{subsec:conduche}
\begin{paragr}
  Let $k\geq 0$ and $C$, $D$ be \ook k-categories. A morphism $f:C\to D$ is a {\em discrete Conduché fibration} if it satisfies the following conditions:
  \begin{enumerate}
  \item For $0\leq n<m$ and  $y\in D_n$,  $x\in C_m$ such that $f(x)=\unit
    m(y)$, there is a unique $z\in C_n$ such that $x=\unit m(z)$.
    \item For $0\leq i<n$, if $y_1$, $y_2$ are $i$-composable cells in
      $D_n$ and $x\in C_n$ such that $f(x)=y_1\comp i y_2$, there is a unique pair $\pair{x_1}{x_2}$ of $i$-composable cells in $C_n$ such that $x=x_1\comp i x_2$, $f(x_1)=y_1$ and $f(x_2)=y_2$.
    \end{enumerate}
  \end{paragr}
  \begin{remark}
    In fact, condition~(2) implies condition~(1), but due to the
    design of the formal expressions of~\ref{ssubsec:formex}, it makes the
    proofs easier when treated separately.
  \end{remark}
\begin{paragr}  
In the case where $C$, $D$ are plain \oo-categories, that is if
$k=\omega$, we know from~\cite[Theorem 6.11]{guetta:poldcf} that if
$f:C\to D$ is a discrete Conduché fibration and $D$ is freely
generated by a polygraph, then so is $C$. We aim to prove that the
result still holds for any $k\in\N$.
  \end{paragr}
  \begin{remark}
     A key feature of a free \oo-category is that its generators are
     exactly the indecomposable cells, hence uniquely determined
     (see~\cite[Section 4, Proposition 8.3]{makkai:worcom}). This is
     no longer true for $\pair{\omega}{k}$-categories when $k<\omega$, which makes the general case a
     priori significantly more complicated. However, the proof
     of~\cite[Theorem 6.11]{guetta:poldcf} does not rely on the
     uniqueness of the basis, and adapts to the general case.
  \end{remark}
    
\subsection{Lifting of free
  $(\omega,k)$-categories}\label{subsec:lifting}
\begin{paragr}
  Let us fix an integer $k\in\N$ and two $\pair{\omega}{k}$-categories
$C$, $D$ such that $D$ is the free $\pair{\omega}{k}$-category
generated by an $\pair{\omega}{k}$-polygraph $Q$, that is
$D=\frgp{Q}$. We prove that if $f:C\to D$ is a discrete Conduché
fibration, then $C$ is itself freely generated by an
$\pair{\omega}{k}$-polygraph $P$. This amounts to prove that, for each
$n\in \N$, the $\pair nk$-category $\trk n(C)$ is freely generated by
a $\pair nk$-polygraph. We reason by induction on $n$. Note that for
$n\leq k$, the result is already proved in~\cite{guetta:poldcf}, so that
we may start the induction at level $k$. The case $n=k$ is already
covered by~\cite{guetta:poldcf}. The induction step is a consequence
of the following fundamental lemma.
\end{paragr}
\subsubsection{Fundamental Lemma}\label{ssubsec:fundamental}
Let $f:C\to D$ be morphism in $\npCat nk$ where $D$ is freely
generated by an $\pair{\omega}{k}$-polygraph $Q$ and $n\geq k$. Let
$T=Q_{n+1}$ be the set of $(n{+}1)$-generators of $Q$ and
$\geninc^D:T\to D_{n+1}$ the insertion of generators.  Let us define
\[
  S=f^{-1}(\geninc^D(T))=\setof{a\in C_{n+1}}{f(a)\in \geninc^D(T)}.
\]
The restriction of $\sce n^C,\tge n^C$ to $S$ determines a
cellular extension $\pair{\trk nC}S$ of the $\pair nk$-category $\trk
nC$. Therefore, by~\cref{prop:univprop}, there is a unique map
\[\phi: \wtxeq S\to D_{n+1}\]
extending $\trk n f$ to a morphism $\extend\phi:\frf{n,k}(\trk n C,S)\to
\trk{n+1}(D)$. By the same proposition, there is a unique map
\[\chi:\wtxeq S\to C_{n+1}\]
extending the identity $\unit{}:\trk nC\to\trk n C$ to a morphism
\[\extend\chi:\frf{n,k}(\trk n C,S)\to
  \trk{n+1}(C).\]
Note that, by construction, the following triangle commutes:
\[
  \xymatrix{C_{n+1} \ar[d]_f& \wtxeq{S}\ar[l]_{\chi}\ar[dl]^{\phi}\\
  D_{n+1}&}
\]
\begin{lemma}\label{lemma:fundamental}
  If $f:C\to D$ is a discrete Conduché fibration, then the morphism $\extend\chi:\frf{n,k}(\trk n C,S)\to\trk{n+1}C$ is an isomorphism.
\end{lemma}