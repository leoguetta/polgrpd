\section{The polygraphic resolution theorem}

\subsection{}
\begin{paragr}
  Let $C$ be a small category and consider the canonical functor
  \[
    \begin{aligned}
      \tr{C}{-} \colon C &\mapsto \Cat \\
      c &\mapsto \tr{C}{c}.
    \end{aligned}
  \]
  % Note that if $C$ is a groupoid, then this functor lands in
  % $\Grpd$.
  Now let
  \[
    P \to C
  \]
  be a polygraphic resolution of $C$ in $\opCat{1}$. For an object $c$
  if $C$, we can define the \ook{1}\nbd-category $\tr{P}{c}$
  as the following pullback
  \[
    \begin{tikzcd}
      \tr{P}{c} \ar[d] \ar[r] & \tr{C}{c} \ar[d] \\
      P \ar[r] & C.
      \ar[from=1-1,to=2-2,phantom,"\lrcorner",very near start]
    \end{tikzcd}
  \]
  This construction extends canonically to a functor
  \[
    \begin{aligned}
      \tr{P}{-} \colon C &\to \opCat{1} \\
      c &\mapsto \tr{P}{c}.
    \end{aligned}
  \]

  Note that in the case that $C=G$ is a groupoid, then the same
  construction applies when taking $P \to G$ to be a polygraphic
  resolution in $\oGrpd=\opCat{0}$. In this case, we obtain a functor
  \[
       \begin{aligned}
      \tr{P}{-} \colon G &\to \opGrpd \\
      x &\mapsto \tr{P}{x}.
    \end{aligned} 
  \]
  with values in $\opGrpd$.
\end{paragr}
\begin{theorem}
  Let $C$ be a (small) category and $P \to C$ a polygraphic resolution
  of $C$ in $\opCat{1}$. 
\end{theorem}