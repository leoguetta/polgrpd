\section{The polygraphic resolution theorem}

\subsection{}
\begin{paragr}
  Let $C$ be a small category and consider the canonical functor
  \[
    \begin{aligned}
      \tr{C}{-} \colon C &\mapsto \Cat \\
      c &\mapsto \tr{C}{c}.
    \end{aligned}
  \]
  % Note that if $C$ is a groupoid, then this functor lands in
  % $\Grpd$.
  Now let
  \[
    P \to C
  \]
  be a polygraphic resolution of $C$ in $\opCat{1}$. For an object $c$
  if $C$, we can define the \ook{1}\nbd-category $\tr{P}{c}$
  as the following pullback
  \[
    \begin{tikzcd}
      \tr{P}{c} \ar[d] \ar[r] & \tr{C}{c} \ar[d] \\
      P \ar[r] & C.
      \ar[from=1-1,to=2-2,phantom,"\lrcorner",very near start]
    \end{tikzcd}
  \]
  This construction extends canonically to a functor
  \[
    \begin{aligned}
      \tr{P}{-} \colon C &\to \opCat{1} \\
      c &\mapsto \tr{P}{c}.
    \end{aligned}
  \]

  % By post-composing with the abelianisation functor
  % \[
  %   \lambda \colon \opCat{1} \to \Chp,
  % \]
  % we obtain a 

  Note that in the case that $C=G$ is a groupoid, then the same
  construction applies when taking $P \to G$ to be a polygraphic
  resolution in $\oGrpd=\opCat{0}$. In this case, we obtain a functor
  \[
       \begin{aligned}
      \tr{P}{-} \colon G &\to \oGrpd \\
      x &\mapsto \tr{P}{x}.
    \end{aligned} 
  \]
  with values in $\oGrpd$.
\end{paragr}
\begin{theorem}
  Let $C$ be a (small) $k$\nbd-category with $k \in \{0,1\}$ (i.e.\
  $C$ is a category or a groupoid),  $i \in \N\cup\{\omega\}$ such that
  $i \geq k$, and $P \to C$ be a polygraphic
  resolution in $\opCat{i}$. The functor
    \[
    \begin{aligned}
      \tr{P}{-} \colon C &\to  \opCat{i}\\
      x &\mapsto \tr{P}{x}
    \end{aligned}
  \]
  is a projective resolution of the terminal object of the category
  $\fun{C}{\opCat{i}}$, equipped with the projective model structure
  induced by the folk model structure on $\opCat{i}$.
\end{theorem}
% \begin{theorem}
%   Let $C$ be a (small) category and $P \to C$ a polygraphic resolution
%   of $C$ in $\opCat{1}$. The functor
%   \[
%     \begin{aligned}
%       \tr{P}{-} \colon C &\to  \opCat{1}\\
%       x &\mapsto \tr{P}{x}
%     \end{aligned}
%   \]
%   is a projective resolution of the terminal object of the category
%   $\fun{C}{\opCat{1}}$, equipped with the projective model structure
%   induced by the folk model structure on $\opCat{1}$.
% \end{theorem}

% \begin{theorem}
%   Let $G$ be a (small) groupoid and $P \to G$ a polygraphic resolution
%   of $G$ in $\oGrpd$. The functor
%   \[
%     \begin{aligned}
%       \tr{P}{-} \colon G &\to  \oGrpd\\
%       x &\mapsto \tr{P}{x}
%     \end{aligned}
%   \]
%   is a projective resolution of the terminal object of the category
%   $\fun{C}{\oGrpd}$, equipped with the projective model structure
%   induced by the folk model structure on $\oGrpd$.
% \end{theorem}
\begin{lemma}
  Let $C$ be a $k$\nbd-category, with $k \in \{0,1\}$ (i.e.\ $C$ is a
  category or a groupoid). For every object $x$ of $C$, the canonical
  functor
  \[
    \tr{C}{x} \to C
  \]
  is a discrete Conduché functor.
\end{lemma}
\begin{proof}
  It suffices to treat the case $k=1$ (i.e.\ $C$ is a category) as the canonical inclusion
  $\Grpd~\to~\Cat$ trivially preserves and reflects Conduché functors.

  Recall that an object of $\tr{C}{x}$ consists of a pair $(y,f \colon
  y \to x)$, where $y$ is an object of $C$ and $f$ an arrow of $C$,
  and a morphism $(y,f) \to (y',f')$ consists of an arrow $g \colon y
  \to y'$ such that the following triangle is commutative
  \[
    \begin{tikzcd}[column sep=small]
      y \ar[rr,"g"] \ar[dr,"f"'] && y'\ar[dl,"f'"] \\
      &x&
    \end{tikzcd}
  \]
  % If $g=\unit y$ (and thus $y=y'$), then clearly it is a unit as an
  % arrow of $\tr{C}{x}$.

  If $g=g_2\circ g_1$, then we have the following decomposition in
  $\tr{C}{x}$,
  \[
      \begin{tikzcd}
      y \ar[rr,"g"] \ar[dr,"f"'] && y'\ar[dl,"f'"] \\
      &x&
    \end{tikzcd}
    =
    \begin{tikzcd}
      y \ar[r,"g_1"] \ar[dr,"f"']& z \ar[r,"g_2"] \ar[d,"h"] &
      y' \ar[dl,"f'"] \\
      &x,&
    \end{tikzcd}
  \]
  with $h=g_2\circ f'$. This choice of $h$ is clearly unique, thus
  proving that $\tr{C}{x} \to C$ is a Conduché functor.
\end{proof}
