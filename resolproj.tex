\section{The polygraphic resolution theorem}

\subsection{Folk model structure}
\begin{paragr}
  Recall that there exists a so-called ``folk'' model category
  structure on the category $\oCat$, characterized by the fact that
  all objects are fibrant and the cofibrations are generated by the
  set
  \[
    \setof{\Sph {n-1} \to \Dsk n }{n \in \N}.
  \]
  The cofibrant objects of this model structure are exactly the
  \oo\nbd-categories free on a polygraph. For a description of the
  weak equivalences see \cite{}. 
\end{paragr}
\begin{paragr}
  More generally, for every $k \in \N$, there exists a model
  category structure on the category $\opCat{k}$ which is
  right-induced from the inclusion functor
  \[
    \opCat{k} \to \oCat.
  \]
  In other words, every object of $\opCat{k}$ is fibrant and, if
  $\trun{k} \colon \oCat \to \opCat{k}$ is the left adjoint of the
  previous inclusion functor, which formally inverts all cells of
  dimension $> k $, then the cofibrations of $\opCat{k}$ are generated
  by the set \lgcomment{On a une référence pour ça?}
  \[
    \setof{\trun{k}\Sph {n-1} \to \trun{k}\Dsk n }{n \in \N}.
  \]
\end{paragr}

\begin{paragr}
  As with any cofibrantly generated model structure, given a (small)
  category $C$, we can consider the \emph{projective} model structure
  on the category of functors
  \[
    \fun{C}{\opCat{k}},
  \]
  for any $k \in \N\cup\{\omega\}$. The weak equivalences, fibrations
  and trivial fibrations are the pointwise ones (in particular, every
  object is fibrant) and the cofibrations are generated by the set of
  morphisms
  \[
    \coprod_{\homset{C}{c}{-}}\Sph{n-1} \to \coprod_{\homset{C}{c}{-}}\Dsk{n},
  \]
  for all object $c$ of $C$ and $n \in \N$.
\end{paragr}
\subsection{The polygraphic resolution theorem}
\begin{paragr}
  Let $C$ be a small category and consider the canonical functor
  \[
    \begin{aligned}
      \tr{C}{-} \colon C &\mapsto \Cat \\
      c &\mapsto \tr{C}{c}.
    \end{aligned}
  \]
  % Note that if $C$ is a groupoid, then this functor lands in
  % $\Grpd$.
  Now let $i \in \N\cup\{\omega\}$ such that $i \geq k$, and let
  \[
    P \to C
  \]
  be a polygraphic resolution of $C$ in $\opCat{i}$. Note that this
  makes sense because we have a canonical inclusion $\Cat \to
  \opCat{i}$. For an object $c$
  if $C$, we can define the \ook{i}\nbd-category $\tr{P}{c}$
  as the following pullback
  \[
    \begin{tikzcd}
      \tr{P}{c} \ar[d] \ar[r] & \tr{C}{c} \ar[d] \\
      P \ar[r] & C.
      \ar[from=1-1,to=2-2,phantom,"\lrcorner",very near start]
    \end{tikzcd}
  \]
  This construction extends canonically to a functor
  \[
    \begin{aligned}
      \tr{P}{-} \colon C &\to \opCat{i} \\
      c &\mapsto \tr{P}{c}.
    \end{aligned}
  \]

  % By post-composing with the abelianisation functor
  % \[
  %   \lambda \colon \opCat{1} \to \Chp,
  % \]
  % we obtain a 

  Note that in the case that $C=G$ is a groupoid, then, since we have
  an inclusion $\Grpd \to \opCat{0}=\oGrpd$, we can take a polygraphic
  resolution $P \to G$  in $\oGrpd$. In this case, we obtain a functor
  \[
       \begin{aligned}
      \tr{P}{-} \colon G &\to \oGrpd \\
      c &\mapsto \tr{P}{c}.
    \end{aligned} 
  \]
\end{paragr}
\begin{theorem}\label{thm:projresol}
  Let $C$ be a (small) $k$\nbd-category with $k \in \{0,1\}$ (i.e.\
  $C$ is a category or a groupoid),  $i \in \N\cup\{\omega\}$ such that
  $i \geq k$, and $P \to C$ be a polygraphic
  resolution in $\opCat{i}$. The functor
    \[
    \begin{aligned}
      \tr{P}{-} \colon C &\to  \opCat{i}\\
      c &\mapsto \tr{P}{c}
    \end{aligned}
  \]
  is a cofibrant object of the category
  $\fun{C}{\opCat{i}}$, equipped with the projective model structure.
\end{theorem}
% \begin{theorem}
%   Let $C$ be a (small) category and $P \to C$ a polygraphic resolution
%   of $C$ in $\opCat{1}$. The functor
%   \[
%     \begin{aligned}
%       \tr{P}{-} \colon C &\to  \opCat{1}\\
%       x &\mapsto \tr{P}{x}
%     \end{aligned}
%   \]
%   is a projective resolution of the terminal object of the category
%   $\fun{C}{\opCat{1}}$, equipped with the projective model structure
%   induced by the folk model structure on $\opCat{1}$.
% \end{theorem}

% \begin{theorem}
%   Let $G$ be a (small) groupoid and $P \to G$ a polygraphic resolution
%   of $G$ in $\oGrpd$. The functor
%   \[
%     \begin{aligned}
%       \tr{P}{-} \colon G &\to  \oGrpd\\
%       x &\mapsto \tr{P}{x}
%     \end{aligned}
%   \]
%   is a projective resolution of the terminal object of the category
%   $\fun{C}{\oGrpd}$, equipped with the projective model structure
%   induced by the folk model structure on $\oGrpd$.
% \end{theorem}
\begin{lemma}\label{lemma:slicescond}
  Let $C$ be a $k$\nbd-category, with $k \in \{0,1\}$ (i.e.\ $C$ is a
  category or a groupoid). For every object $c$ of $C$, the canonical
  functor
  \[
    \tr{C}{c} \to C
  \]
  is a discrete Conduché functor.
\end{lemma}
\begin{proof}
  It suffices to treat the case $k=1$ (i.e.\ $C$ is a category) as the canonical inclusion
  $\Grpd~\to~\Cat$ trivially preserves and reflects Conduché functors.

  Recall that an object of $\tr{C}{x}$ consists of a pair $(y,f \colon
  y \to x)$, where $y$ is an object of $C$ and $f$ an arrow of $C$,
  and a morphism $(y,f) \to (y',f')$ consists of an arrow $g \colon y
  \to y'$ such that the following triangle is commutative
  \[
    \begin{tikzcd}[column sep=small]
      y \ar[rr,"g"] \ar[dr,"f"'] && y'\ar[dl,"f'"] \\
      &x&
    \end{tikzcd}
  \]
  % If $g=\unit y$ (and thus $y=y'$), then clearly it is a unit as an
  % arrow of $\tr{C}{x}$.

  If $g=g_2\circ g_1$, then we have the following decomposition in
  $\tr{C}{x}$,
  \[
      \begin{tikzcd}
      y \ar[rr,"g"] \ar[dr,"f"'] && y'\ar[dl,"f'"] \\
      &x&
    \end{tikzcd}
    =
    \begin{tikzcd}
      y \ar[r,"g_1"] \ar[dr,"f"']& z \ar[r,"g_2"] \ar[d,"h"] &
      y' \ar[dl,"f'"] \\
      &x,&
    \end{tikzcd}
  \]
  with $h=g_2\circ f'$. This choice of $h$ is clearly unique, thus
  proving that $\tr{C}{x} \to C$ is a Conduché functor.
\end{proof}
\begin{proof}[Proof of \cref{thm:projresol}]
By \cref{lemma:slicescond} and the fact that the class of discrete Conduché functors is
stable by pullback, it follows that for every object $c$ of $C$, the
morphism
\[
  \tr{P}{c} \to P
\]
is a discrete Conduché functor. By , it follows that
$\tr{P}{c}$ is a polygraph in $\opCat{i}$. More precisely, let
$\Sigma_n$ be a set of generators of dimension $n$ of $P$, and let us
denote by $\tr{\Sigma_n}{c}$ the set
\[
  \tr{\Sigma_n}{c}=\setof{(x,f)}{x \in \Sigma_n, f : \target_0(x) \to c
    \text{ in }C}.
\]
By \ref{}, we know that $\tr{\Sigma_n}{c}$ is a set of generators of
dimension $n$ of the polygraph $\tr{P}{c}$. In particular, this means
that we have a cocartesian square
\[
  \begin{tikzcd}[column sep=huge]
   \displaystyle \coprod_{x \in \Sigma_n}\coprod_{f \in
     \homset{C}{\target_0(x)}{c}}\Sph{n-1} \ar[d]\ar[r,"{\langle \source(x,f),\target(x,g) \rangle}"]& \trk{n-1}\tr{P}{c} \ar[d]\\
    \displaystyle \coprod_{x \in \Sigma_n}\coprod_{f \in
     \homset{C}{\target_0(x)}{c}}\Dsk{n}\ar[r,"{\langle x,f \rangle}"'] & \trk{n}\tr{P}{c}
\ar[from=1-2,to=2-2,phantom,"\ulcorner", very near end]
 \end{tikzcd}
\]
It is straighforward to check that this square is natural in $c$, and
thus, by definition of the projective fibrations, this proves that
$\trk{n-1}\tr{P}{-} \to \trk{n}\tr{P}{-}$ is a projective
cofibration. The conclusion follows from the stability of cofibration
by transfinite composition.
\lgcomment{Réécrire un peu cette preuve}
\end{proof}


