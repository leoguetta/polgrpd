\section{Homology of categories as Tor}
\subsection{Category of modules}
\begin{definition}
  Let $C$ be a (small) category. A \emph{right $C$\nbd-module} is a
  functor
  \[
    F \colon C^{\op} \to \Ab.
  \]
  A \emph{morphism of right $C$\nbd-modules} is a natural
  transformation. We denote by $\RMod{C}$ the category of right
  $C$\nbd-modules. Dually, a \emph{left $C$\nbd-module} is a functor
  \[
    C \to \Ab.
  \]
  The category of left $C$\nbd-modules is denoted by $\LMod{C}$.
\end{definition}
\begin{example}
  If $C=G$ is a group, then left and right $C$\nbd-modules
  respectively correspond
  exactly to left and right $\Z[G]$\nbd-modules. 
\end{example}

As we shall now see, the abelian categories $\RMod{C}$ and $\LMod{C}$
have enough projectives. Since $\LMod{C}=\RMod{C^{\op}}$, it suffices
to treat the case of right modules.

\begin{lemma}\label{lemma:freeisproj}
  Let $c$ be an object of $C$. The right $C$\nbd-module $\Z[\homset{C}{-}{c}]$
  \[
    \begin{aligned}
      C^{\op} &\to \Ab\\
      d &\mapsto \Z[\homset{C}{d}{c}],
    \end{aligned}
  \]
  is a projective object of $\RMod{C}$.
\end{lemma}
\begin{proof}
  This follows straightforwardly from the Yoneda lemma.
\end{proof}
\begin{proposition}\label{prop:enoughproj}
  The category $\RMod{C}$ has enough projectives.
\end{proposition}
\begin{proof}
  Let $F$ be an object of $\RMod{C}$. Let us denote by $\el{F}$ the
  category of elements of $F$ when discarding the abelian group
  structures. Consider the canonical
  morphism
  \[
    \bigoplus_{\homset{C}{-}{c} \to F \in \el{F}} \Z[\homset{C}{-}{c}] \to F.
  \]
 It follows from the Yoneda lemma that this morphism is
  an epimorphism. Since coproducts of projective objects
  are projective, the conclusion follows from \cref{lemma:freeisproj}.
\end{proof}
\subsection{Tensor product}
We fix once and for all a small category $C$.
\begin{paragr}
   Let $F$ be a right $C$\nbd-module and $G$ a left $C$\nbd-module. We
  define their tensor product, denoted by $F\tens{C}G$, as the
  following co-end in $\Ab$
  \[
    F\tens{C}G := \int^{c \in C}F(c)\otimes G(c),
  \]
  where $\otimes$ is the usual tensor product of abelian
  groups. Explicitly, this means that $F\tens{C}G$ is obtained as the following
  co-equalizer in $\Ab$
  \[
    \coprod_{(c,c')\in \Ob(C)^2}F(c')\otimes\Z[\homset{C}{c}{c'}]\otimes
    G(c) \rightrightarrows \coprod_{c \in \Ob(C)}F(c)\otimes G(c)
    \rightarrow F\tens{C}G,
  \]
  where the top arrow is induced by the functoriality of $F$ and the
  bottow arrow by the functoriality of $G$. This construction canonically extends to a functor
  \[
    - \tens{C} - \colon \RMod{C}\times\LMod{C} \to \Ab.
  \]
  It is straightforward to check that this functor is a left adjoint
  in each variable. More precisely, we have isomorphisms
  \[
    \begin{aligned}
      \homset{\Ab}{F\tens{C}G}{A}&\simeq
                                  \homset{\RMod{C}}{F}{\homset{\Ab}{G(-)}{A}}\\
      &\simeq \homset{\LMod{C}}{G}{\homset{\Ab}{F(-),A}}
    \end{aligned}
  \]
  natural in $F$, $G$ and $A$. In particular, $-\tens{C}-$ preserves
  epimorphisms in each variable and thus is right exact in each
  variable. Together with \cref{prop:enoughproj}, this justifies the
  following definition.
\end{paragr}

\begin{definition}
  Let $F$ be a right $C$\nbd-module and $G$ a left $C$\nbd-module. For
  an integer $k \geq 0$, we
  define the abelian group $\Tor{C}{k}(F,G)$ as
  \[
    \Tor{C}{k}(F,G) := F \overset{\mathbb{L}_k}{\tens{C}} G,
  \]
  where $\overset{\mathbb{L}_k}{\tens{C}}$ is the $k$\nbd-th left
  derived functor of $-\tens{C}-$.
\end{definition}
\begin{paragr}
  Explicitly, the previous definition means that $\Tor{C}{k}(F,G)$ is
  obtained by either taking a projective resolution
  \[
    P_{\bullet} \to F
  \]
  of $F$ (in $\RMod{C}$), or a projective resolution
  \[
    Q_{\bullet} \to G
  \]
  of $G$ (in $\LMod{C}$) and taking the $k$\nbd-th homology group of
  either the chain complex of abelian groups
  \[P_{\bullet}\tens{C}G\]
  or the chain complex of abelian groups
  \[F\tens{C}Q_{\bullet}\] (both yielding the same result by standard
  results of homological algebra).
\end{paragr}
\subsection{Homology}
\begin{paragr}
  For a (small) category $C$, we denote indifferently by $\cst{\Z}$
  the trivial left and right $C$-module with value $\Z$, that is to
  say, the constant functors
  \[
    C^{\op} \to \Ab
  \]
  and
  \[
    C \to \Ab
  \]
  with value $\Z$.
\end{paragr}
\begin{definition}
  Let $C$ be a small category. We define its \emph{homology $\Ho_{\bullet}(C)$} as the
  graded abelian group
  \[
    \Ho_{\bullet}(C):=\Tor{C}{\bullet}(\Z,\Z).
  \]
  More generally, if $M$ is a right $C$\nbd-module, we define the
  \emph{homology $\Ho_{\bullet}(C,M)$ of $C$ with coefficient $M$} as the
  graded abelian group
  \[
    \Ho_{\bullet}(C,M):=\Tor{C}{\bullet}(M,\Z).
  \]
\end{definition}
\begin{remark}
  We could also have defined the homology of $C$ with coefficient a left
  $C$\nbd-module $N$ as $\Tor{C}{\bullet}(\Z,N)$. No generality is
  lost as we
  have \[\Tor{C}{\bullet}(\Z,N)=\Tor{C^{\op}}{\bullet}(N,\Z).\]
  In other words, the homology of $C$ with coefficient a left
  $C$\nbd-module $N$ is the homology of $C^{\op}$ with coefficient $N$
  seen a right $C^{\op}$\nbd-module.
\end{remark}