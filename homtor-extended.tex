\documentclass{amsart}

\usepackage{amssymb}
\usepackage{amsthm}
\usepackage{amsmath}

\usepackage{tikz-cd}
\usepackage{leftindex}
\usepackage[capitalise]{cleveref}
\usepackage[all,2cell]{xy}
\UseAllTwocells
\usepackage{bm} %bold math symbols

\usepackage{marginnote}

\title{Homology coefficient in a natural system}

%ENVIRONMENTS

\swapnumbers
\theoremstyle{plain}
\newtheorem{theorem}{Theorem}[subsection]
\newtheorem{corollary}[theorem]{Corollary}
\newtheorem{proposition}[theorem]{Proposition}
\newtheorem{lemma}[theorem]{Lemma}

\theoremstyle{definition}
\newtheorem{definition}[theorem]{Definition}
\newtheorem{example}[theorem]{Example}
\newtheorem{paragr}[theorem]{} %Environnement "paragraphe"

\theoremstyle{remark}
\newtheorem{remark}[theorem]{Remark}

% Cleveref definitions
\crefname{theorem}{Theorem}{Theorems}
\crefname{corollary}{Corollary}{Corollaries}
\crefname{proposition}{Proposition}{Propositions}
\crefname{lemma}{Lemma}{Lemmas}

\crefname{definition}{Definition}{Definitions}
\crefname{example}{Example}{Examples}

\crefname{rem}{Remark}{Remarks}
\crefname{paragr}{}{}

% MATH G

\newcommand{\nmbr}[1]{\mathbb #1}
\newcommand{\N}{\nmbr{N}}
\newcommand{\Z}{\nmbr{Z}}
\newcommand{\Q}{\nmbr{Q}}
\newcommand{\R}{\nmbr{R}}
\newcommand{\C}{\nmbr{C}}

\newcommand{\set}[1]{\left\{#1\right\}} 
\newcommand{\setof}[2]{\left\{ #1\; | \; #2\right\}}
\newcommand{\pair}[2]{(#1,#2)}       %pair

% CATEGORIES

% general

\newcommand{\psh}[1]{\widehat{#1}} % presheaves
\newcommand{\plim}{\varprojlim} % limit
\newcommand{\ilim}{\varinjlim} % colimit
\newcommand{\source}{\mathsf{s}} % source
\newcommand{\target}{\mathsf{t}} % target
\newcommand{\cosource}{\sigma} % source
\newcommand{\cotarget}{\tau} % target
\newcommand{\geninc}{\mathsf{i}} % inclusion of generators
\newcommand{\cmp}{\circ} % composition
\newcommand{\unit}[1]{1^{#1}} % unit
\newcommand{\homset}[3]{#1(#2,#3)} % homset in #1 from #2 to #3
\newcommand{\fun}[2]{\mathbf{Hom}(#1,#2)} %Functor categories
\newcommand{\Ob}{\mathrm{Ob}} %Set of objects
\newcommand{\Mor}{\mathrm{Mor}} %Set of morphisms
\newcommand{\prd}[1]{\underset{#1}{\times}} 
\newcommand{\sumalg}[1]{\underset{#1}{+}} 
\newcommand{\pbck}[3]{#1\prd{#3} #2} % pullback #1 x_#3 #2
\newcommand{\psht}[3]{#1\sumalg{#3} #2} % pushout #1 +_#3 #2
\newcommand{\op}{{\mathrm{op}}} % opposite category symbol
\newcommand{\el}[1]{\mathrm{el}(#1)} %Category of elements of a presheaf
\newcommand{\tr}[2]{#1/#2} %Slice categories
\newcommand{\cotr}[2]{#2\backslash #1} %coslice categories
% special categories

\newcommand{\ctg}[1]{\mathbf{#1}}  % large category
\newcommand{\Set}{\ctg{Set}}    % category of sets
\newcommand{\Cat}{\ctg{Cat}}   % category of small categories
\newcommand{\nCat}[1]{\Cat_{#1}}  % n-categories
\newcommand{\npCat}[2]{\Cat_{#1,#2}}  % (n,p)-categories
\newcommand{\opCat}[1]{\Cat_{\omega,#1}} % (omega,p)-categories
\newcommand{\nCatp}[1]{\Cat_{#1}^{+}} % cellular extensions
\newcommand{\oCat}{\nCat{\omega}} % omega-categories
\newcommand{\glob}{\ctg{G}}% site of globular sets
\newcommand{\Glob}{\ctg{Glob}} 
\newcommand{\nGlob}[1]{\Glob_{#1}} % n-globular sets
\newcommand{\oGlob}{\nGlob{\omega}} % globular sets
\newcommand{\Pol}{\ctg{Pol}} 
\newcommand{\nPol}[1]{\Pol_{#1}} % n-polygraphs
\newcommand{\oPol}{\nPol{\omega}} % polygraphs
\newcommand{\npPol}[2]{\Pol_{#1,#2}} % (#1,#2)-polygraphs
\newcommand{\opPol}[1]{\npPol{\omega,#1}} % (omega,#1)-polygraphs
\newcommand{\Grpd}{\ctg{Grpd}}
\newcommand{\nGrpd}[1]{\Grpd_{#1}} % n-groupoids
\newcommand{\oGrpd}{\nGrpd{\omega}} %omega-groupoids
\newcommand{\Ch}[1]{\ctg{Ch}_{#1}} % chain complexes
\newcommand{\Chp}{\Ch{\geq 0}} % positive chaincomplexes
\newcommand{\Ab}{\ctg{Ab}} % abelian groups
\newcommand{\Term}{\ctg{1}} % terminal category
\newcommand{\Grph}{\ctg{Grph}}  % graphs
\newcommand{\Mon}{\ctg{Mon}} %  monoids
\newcommand{\Spl}{\Delta} % simplicial category
\newcommand{\SSet}{\psh{\Spl}} % simplicial sets

\newcommand{\RMod}[1]{\mathrm{Mod}_{#1}} %category of right modules
\newcommand{\LMod}[1]{{}_{#1}\mathrm{Mod}} %category of left modules (un peu dégueu...)

% FUNCTORS
\newcommand{\functor}[1]{\mathsf{#1}} % style for functor names
\newcommand{\trk}[1]{\functor{tr}_{#1}}%truncation functor to dim #1
\newcommand{\inc}[1]{\functor{inc}_{#1}}%inclusion functor from dim #1
\newcommand{\fgf}[1]{\functor{U}_{#1}} %forgetful functor
\newcommand{\frf}[1]{\functor{F}_{#1}} %free functor
% ARROWS

\newcommand{\doubl}[2]{\ar@<2pt>[l]^{#2}\ar@<-2pt>[l]_{#1}}
\newcommand{\dbl}{\doubl{}{}}
\newcommand{\doubr}[2]{\ar@<2pt>[r]^{#2}\ar@<-2pt>[r]_{#1}}
\newcommand{\dbr}{\doubr{}{}}
\newcommand{\doubld}[2]{\ar@<2pt>[ld]^{#2}\ar@<-2pt>[ld]_{#1}}
\newcommand{\dbld}{\doubld{}{}}


% OMEGA-CAT BASICS

\newcommand{\comp}[1]{\ast_{#1}} % composition along dim #1
\newcommand{\sce}[1]{\source_{#1}} % source of dim #1
\newcommand{\tge}[1]{\target_{#1}} % target of dim #1
\newcommand{\cosce}[1]{\cosource_{#1}} % cosource from dim #1
\newcommand{\cotge}[1]{\cotarget_{#1}} % cotarget from dim #1
\newcommand{\frcat}[1]{#1^*} % free category generated by a polygraph
\newcommand{\frgp}[1]{#1^\top} % free groupoid generated by a polygraph

 \newcommand{\oo}{$\omega$}
 \newcommand{\ook}[1]{$\pair{\omega}{#1}$}
\newcommand{\clx}[2]{\pair{#1}{#2}}% cellular extension of #1 by #2
\newcommand{\gni}[1]{\geninc_{#1}}% inclusion of #1-generators

\newcommand{\Sph}[1]{\mathbb{S}^{#1}} %n-sphere
\newcommand{\Dsk}[1]{\mathbb{D}_{#1}} %n-disk
\newcommand{\trun}[1]{\tau_{#1}} %truncation functor

% FORMAL EXPRESSIONS

\newcommand{\expr}[1]{\mathcal{E}[#1]} %formal expressions on #1
\newcommand{\wtx}[1]{\mathcal{W}[#1]} % well typed expressions on #1
\newcommand{\formal}[1]{\mathsf{#1}} % formal expressions style
\newcommand{\fcst}[1]{\formal{c}_{#1}} % constant
\newcommand{\finv}[1]{\overline{\formal{c}}_{#1}}% formal inverse
\newcommand{\fid}[1]{\formal{i}_{#1}} % formal identity
\newcommand{\fcomp}[1]{\bm{\ast}_{#1}} % formal composition
\newcommand{\type}[3]{#1:#2\to #3} %typing rule

                              
% HOMOLOGY AND HOMOLOGICAL ALGEBRA

 \newcommand{\tens}[1]{\otimes_{#1}} %tensor product
\newcommand{\Tor}[2]{\functor{Tor}^{#1}_{#2}} %Tor
\newcommand{\cst}[1]{\underline{#1}} %Constant module
\newcommand{\Ho}{\functor{H}} %Homology


%OTHERS

\newcommand{\nbd}{\nobreakdash} %Shortcut

%COMMENTS

\newcommand{\lgcomment}[1]{
    \marginnote{\tiny\textcolor{red}{LG: #1}}
}

\newcommand{\M}{\mathcal{M}}
\newcommand{\B}{\mathcal{B}}
\newcommand{\Bimod}[1]{\mathrm{Bimod}_{#1}}
\newcommand{\Tw}[1]{\mathrm{Tw}(#1)}
\newcommand{\Nat}[1]{\mathrm{Nat}(#1)}
\newcommand{\HHo}{\mathsf{HHo}}
\newcommand{\Tot}{\mathrm{Tot}}
\newcommand{\id}{\mathrm{id}}
\newcommand{\src}[1]{\source{} #1}
\newcommand{\tgt}[1]{\target{} #1}
\renewcommand{\sce}[1]{\source{} #1}
\renewcommand{\tge}[1]{\target{} #1}
\newcommand{\srclin}{\sigma}
\newcommand{\tgtlin}{\tau}
\renewcommand{\homset}[3]{\mathrm{Hom}_{#1}(#2,#3)}
\newcommand{\wght}[1]{\mathrm{w}_{#1}}
\begin{document}
\section{Homological algebra}
\subsection{Category of modules}
\begin{definition}
  Let $C$ be a (small) category. A \emph{right $C$\nbd-module} is a
  functor
  \[
    F \colon C^{\op} \to \Ab.
  \]
  A \emph{morphism of right $C$\nbd-modules} is a natural
  transformation. We denote by $\RMod{C}$ the category of right
  $C$\nbd-modules. Dually, a \emph{left $C$\nbd-module} is a functor
  \[
    C \to \Ab.
  \]
  The category of left $C$\nbd-module is denoted by $\LMod{C}$.
\end{definition}
\begin{example}
  If $C=G$ is a group, then left and right $C$\nbd-modules
  respectively correspond
  exactly to left and right $\Z[G]$\nbd-modules. 
\end{example}

As we shall now see, the abelian categories $\RMod{C}$ and $\LMod{C}$
have enough projectives. Since $\LMod{C}=\RMod{C^{\op}}$, it suffices
to treat the case of right modules.

\begin{definition}
  A right $C$\nbd-modules $M$ is free if it is isomorphic to a small
  sum
  of ``representables'' modules
  \[
  M \simeq \bigoplus_{i \in I}\Z\homset{C}{-}{c_i},
\]
and dually for left $C$\nbd-modules.
\end{definition}
\begin{lemma}\label{lemma:freeisproj}
  Free modules are projective objects of $\RMod{C}$.
\end{lemma}
\begin{proof}
  This follows straightforwardly from the Yoneda lemma and the fact
  that a sum of projective objects is projective.
\end{proof}
\begin{proposition}\label{prop:enoughproj}
  The category $\RMod{C}$ has enough free objects (and thus enough projectives).
\end{proposition}
\begin{proof}
  Let $F$ be an object of $\RMod{C}$. Consider the canonical
  morphism
  \[
    \bigoplus_{\substack{c \in \Ob(C)\\ x \in F(c)}} \Z\homset{C}{-}{c} \to F.
  \]
 It follows from the Yoneda lemma that this morphism is
  an epimorphism. The conclusion follows from \cref{lemma:freeisproj}.
\end{proof}
\subsection{Total tensor product}
We fix once and for all a small category $C$.
\begin{paragr}
   Let $F$ be a right $C$\nbd-module and $G$ a left $C$\nbd-module. We
  define their total tensor product, denoted by $F\tens{\Z C}G$, as the
  following co-end in $\Ab$
  \[
    F\tens{\Z C}G := \int^{c \in C}F(c)\tens{\Z} G(c),
  \]
  where $\otimes$ is the usual tensor product of abelian
  groups. Explicitly, this means that $F\tens{\Z C}G$ is obtained as the following
  co-equalizer in $\Ab$
  \[
    \bigoplus_{(c,c')\in \Ob(C)^2}F(c')\tens{\Z}\Z\homset{C}{c}{c'}\tens{\Z}
    G(c) \rightrightarrows \bigoplus_{c \in \Ob(C)}F(c)\tens{\Z} G(c)
    \rightarrow F\tens{\Z C}G,
  \]
  where the top arrow is induced by the functoriality of $F$ and the
  bottow arrow by the functoriality of $G$. This construction canonically extends to a functor
  \[
    - \tens{\Z C} - \colon \RMod{C}\times\LMod{C} \to \Ab.
  \]
  It is straightforward to check that this functor is a left adjoint
  in each variable. More precisely, we have isomorphisms
  \[
    \begin{aligned}
      \homset{\Ab}{F\tens{\Z C}G}{A}&\simeq
                                  \homset{\RMod{C}}{F}{\homset{\Ab}{G(-)}{A}}\\
      &\simeq \homset{\LMod{C}}{G}{\homset{\Ab}{F(-),A}}
    \end{aligned}
  \]
  natural in $F$, $G$ and $A$. In particular, $-\tens{\Z C}-$ preserves
  epimorphisms in each variable and thus is right exact in each
  variable. Together with \cref{prop:enoughproj}, this justifies the
  following definition.
\end{paragr}

\begin{definition}
  Let $F$ be a right $C$\nbd-module and $G$ a left $C$\nbd-module. For
  an integer $k \geq 0$, we
  define the abelian group $\Tor{\Z C}{k}(F,G)$ as
  \[
    \Tor{\Z C}{k}(F,G) := F \overset{\mathbb{L}_k}{\tens{\Z C}} G,
  \]
  where $\overset{\mathbb{L}_k}{\tens{\Z C}}$ is the $k$\nbd-th left
  derived functor of $-\tens{\Z C}-$.
\end{definition}
\begin{paragr}
  Explicitly, the previous definition means that $\Tor{\Z C}{k}(F,G)$ is
  obtained by either taking a projective resolution
  \[
    P_{\bullet} \to F
  \]
  of $F$ (in $\RMod{C}$), or a projective resolution
  \[
    Q_{\bullet} \to G
  \]
  of $G$ (in $\LMod{C}$) and taking the $k$\nbd-th homology group of
  either the following chain complexes of abelian groups
  \[P_{\bullet}\tens{\Z C}G,\]
  \[F\tens{\Z C}Q_{\bullet},\]
    \[\Tot(P_{\bullet}\tens{\Z C}Q_{\bullet}). \]
  By standard results of homological algebra, these three complexes
  have the same homology. 
\end{paragr}
\begin{paragr}
  Let $u \colon C \to D$ be a functor between small categories.  As usual, we obtain an adjoint triple
  \[
    \begin{tikzcd}
      \LMod{D} \ar[r,"u^*" description] &
      \LMod{C},\ar[l,"u_!"',bend right] \ar[l,"u_*",bend left]
    \end{tikzcd}
  \]
  where $u^*$ is the pre-composition functor. (And similarly for
  categories of right modules). By standard manipulation of
  homological algebra, one shows that
  \begin{itemize}
  \item $u_!$ is right exact, $u_*$ is left exact, $u^*$ is exact,
  \item $u^*$ preserves injectives and projectives.
  \end{itemize}

  Recall now that a functor $\pi \colon C \to D$ is a discrete
  op-fibration if for every arrow of $D$ of the form $g\colon \pi(c) \to
  d$, there exists a \emph{unique} arrow $f \colon c \to c'$ such
  that $\pi(f)=g$. 
\end{paragr}
\begin{proposition}\label{prop:toropfib}
  Let $\pi \colon C \to D$ a discrete op-fibration, $M$ a right
  $D$\nbd-module and $N$ a left $C$\nbd-module. We have an
  isomorphism
  \[
  \Tor{\Z C}{\bullet}(\pi^*M,N) \simeq \Tor{\Z D}{\bullet}(M,\pi_!N),
  \]
  natural in $M$ and $N$.
\end{proposition}
\begin{proof}
  Let us start by proving the underived version of this isomorphism,
  that is
  \[
   \pi^*M\tens{\Z C} N \simeq M\tens{\Z D}\pi_!N.
 \]
 First, recall that, because $\pi$ is a discrete op-fibration, for every
 object $d$ of $D$, the
 canonical inclusion
 \[
   \begin{aligned}
     \pi_d &\to \tr{C}{d}\\
     c &\mapsto (c,\id_{d} \colon \pi(c) \to d)
   \end{aligned}
 \]
 where $\pi_d$ is the fiber of $\pi$ at $d$, is a right adjoint (this
 is true for non-necessarily discrete op-fibration). In particular,
 this canonical inclusion is co-final and it follows that for every
 object $d$ of $D$, we have
 \[
   (\pi_!N)(d) = \ilim_{f \colon \pi(c) \to d \in \tr{C}{d}}N(c) \simeq
   \bigoplus_{c \in \pi_d}N(c).
 \]
 Second, since

 Now, by definition $M\tens{\Z D}\pi_!N$ is the co-equalizer of the
 following pair
 \[
   \bigoplus_{(d,d') \in
     \Ob(D)^2}M(d')\tens{\Z}\Z\homset{D}{d}{d'}\tens{\Z}(\pi_!N)(d)
   \rightrightarrows \bigoplus_{d \in \Ob(D)}M(d)\tens{\Z}(\pi_!N)(d).
 \]
 It follows from what we have seen above that this pair can be
 re-written as
 \[
   \bigoplus_{(d,d') \in
     \Ob(D)^2}\bigoplus_{c \in
     \pi_d}M(d')\tens{\Z}\Z\homset{D}{d}{d'}\tens{\Z}N(c)
   \rightrightarrows \bigoplus_{d \in \Ob(D)}\bigoplus_{c \in
     \pi_d}M(d)\tens{\Z}N(c).
 \]
 Using the fact that \[\Ob(C)\simeq\coprod_{d \in \Ob(D)}\pi_d,\] and that
 \[
   \coprod_{(c,c')\in \Ob(C)^2}\homset{C}{c}{c'} \simeq
   \coprod_{(d,d') \in \Ob(D)^2}\coprod_{c\in \pi_d}\homset{D}{d}{d'},
 \]
 where this last isomorphism follows from the fact that $\pi$ is a
 discrete op-fibration, we can re-write the pair as
 \[
   \bigoplus_{(c,c') \in
     \Ob(C)^2}M(\pi(c'))\tens{\Z}\Z\homset{C}{c}{c'}\tens{\Z}N(c)\rightrightarrows
   \bigoplus_{c \in \Ob(C)}M(\pi(c))\tens{\Z}N(c),
 \]
 whose co-equalizer is, by definition, $\pi^*M\tens{\Z C}N$.

 For the derived version, it suffices to use that $\pi^*$ is exact and
 preserve projective objects. Hence, if $P_{\bullet} \to M$ is a projective
 resolution of $M$, then $\pi^*P_{\bullet} \to \pi^*M$ is a projective
 resolution of $\pi^*M$. It follows that
 \[
   \pi^*P_{\bullet}\tens{\Z D}N \simeq P_{\bullet}\tens{\Z C}\pi_!N,
 \]
 which proves the result.
\end{proof}
\subsection{Bi-modules and natural systems}
\begin{definition}
  Let $C$ be a (small) category. A \emph{($C,C$)\nbd-bimodule}, or
  \emph{bimodule over $C$}, is a left $C^{\op} \times C$\nbd-module,
  i.e.\ a functor $\B \colon C^{\op} \times C \to \Ab$.

  \end{definition}
  We denote by
  $\Bimod{C}=\LMod{C^{\op}\times C}$ the category of bimodules.
\begin{paragr}
  Given a right $C$\nbd-module $M$ and a left $C$\nbd-module $N$, we
  define the bimodule $M\tens{\Z}N$ as 
  \[
    (c,d) \mapsto M(c)\tens{\Z} N(d).
  \]
  This construction is obviously functorial and thus yields a functor
  \[
    \begin{aligned}
      \RMod{C}\times \LMod{C} \to \Bimod{C}\\
      (M,N) \mapsto M\tens{\Z} N.
    \end{aligned}
  \]
  It is straightforward to check that this functor is a left adjoint
  in both variable. In particular, it preserves epimorphisms in each
  variable and thus is right exact in each variable. 
\end{paragr}
\begin{lemma}\label{tensfreemod}
  The functor $-\tens{\Z}- \colon\RMod{C}\times \LMod{C} \to
  \Bimod{C}$ sends free modules to free modules.
\end{lemma}
\begin{proof}
  It suffices to notice that
  \[
    \begin{aligned}
      \Z\homset{C}{-}{c}\tens{\Z}\Z\homset{C}{d}{-}&=\Z\homset{C^{\op}}{c}{-}\tens{\Z}\Z\homset{C}{d}{-}\\
                                                   &\simeq\Z\left[\homset{C^{\op}}{c}{-}\times\homset{C}{d}{-}\right]\\
      &=\Z\homset{C^{\op}\times C}{(c,d)}{(-,-)}.      
    \end{aligned}
  \]
  The conclusion follows from the fact that $-\tens{\Z}-$ preserves
  sums in each variable. 
\end{proof}
\begin{paragr}
  Notice that every category $C$ comes with a canonical bi-module
  given by the hom functor
  \[
    \Z \homset{C}{-}{-}.
  \]
  This bi-module will play an important role later.
\end{paragr}
\begin{paragr}
  Recall that the \emph{twisted arrow category} (also called the
  factorization category), denoted by $\Tw{C}$, is the category whose
  objects are morphisms $w \colon c \to d$ of $C$, and whose morphisms
  $w \to w'$ are given by pairs $(f\colon c' \to c,g\colon d \to d')$
  of $C$ such that the following square is commutative
  \[
    \begin{tikzcd}
      c \ar[r,"w"] & d \ar[d,"g"] \\
      c' \ar[u,"f"] \ar[r,"w'"] & d'.
    \end{tikzcd}
  \]
\end{paragr}
\begin{definition}
  A \emph{natural system} over a category $C$ is a right $\Tw{C}$
  module,i.e.\ a functor $\M \colon (\Tw{C})^{\op} \to \Ab$.
\end{definition}
We denote by $\Nat{C}=\RMod{\Tw{C}}$ the category of natural systems on $C$.
\subsection{Homology with coefficient}
\begin{paragr}
  For any (small) category $D$ and an abelian group $A$ we will denote by $\cst{A}$ the
  constant functor $D \to \Ab$ with value $A$.%  In particular, for a
  % category $C$, by taking $D$ to be either of the following category
  % \[
  % C, \quad C^{\op}, \quad
  % C^{\op}\times C, \quad \Tw{C},\] $\cst{A}$ can denote, depending on the context, a left $C$\nbd-module, a right
  % $C$\nbd-module, a bimodule over $C$ or a natural system over $C$.
\end{paragr}

\begin{definition}
  Let $C$ be a small category. The \emph{Baues-Wirsching homology of
    $C$ with coefficient in a natural system $\M$} is defined as
  \[
    \Ho^{BS}_{\bullet}(C,\M):=\Tor{\Z \Tw{C}}{\bullet}(\M,\cst{\Z}).
  \]
  The \emph{Hochschild homology of $C$ with coefficient in a bimodule
    $\B$ over $C$} is defined as
  \[
    \HHo_{\bullet}(C,\B):=\Tor{\Z (C^{\op}\times C)}{\bullet}(\B,\Z \homset{C}{-}{-}).
  \]
  The \emph{homology of $C$ with coefficient in a right (resp.\ 
    left) module $M$ over $C$} is defined as
  \[
    \Ho_{\bullet}(C,M):=\Tor{\Z C}{\bullet}(M,\cst{\Z})\quad (\text{resp.\
    } :=\Tor{\Z C}{\bullet}(\cst{\Z},M) ).
  \]
\end{definition}
% \begin{definition}
%   Let $C$ be a small category. We define its \emph{homology $\Ho_{\bullet}(C)$} as the
%   graded abelian group
%   \[
%     \Ho_{\bullet}(C):=\Tor{C}{\bullet}(\Z,\Z).
%   \]
%   More generally, if $M$ is a right $C$\nbd-module, we define the
%   \emph{homology $\Ho_{\bullet}(C,M)$ of $C$ with coefficient $M$} as the
%   graded abelian group
%   \[
%     \Ho_{\bullet}(C,M):=\Tor{C}{\bullet}(M,\Z).
%   \]
% \end{definition}
% \begin{remark}
%   We could also have defined the homology of $C$ with coefficient a left
%   $C$\nbd-module $N$ as $\Tor{C}{\bullet}(\Z,N)$. No generality is
%   lost as we
%   have \[\Tor{C}{\bullet}(\Z,N)=\Tor{C^{\op}}{\bullet}(N,\Z).\]
%   In other words, the homology of $C$ with coefficient a left
%   $C$\nbd-module $N$ is the homology of $C^{\op}$ with coefficient $N$
%   seen a right $C^{\op}$\nbd-module.
% \end{remark}

We are now going to compare these different homologies. Let's start
with the Baues-Wirsching and Hochschild homologies.

\begin{paragr}
  Let $\pi \colon \Tw{C} \to C^{\op}\times C$ be the canonical
  projection. This is a discrete opfibration, and it is
  straightforward to check that
  \[
    \pi_!(\cst{\Z})\simeq \Z\homset{C}{-}{-}.
  \]
\end{paragr}
Applying \cref{prop:toropfib}, we obtain the following proposition.
\begin{proposition}
  Let $C$ be a small category and $\B$ a bimodule over $C$, then we
  have a canonical isomorphism
  \[
    \Ho_{\bullet}^{BS}(C,\pi^*\B)\simeq \HHo_{\bullet}(C,\B).
  \]
\end{proposition}

Let us now compare Hochschild homology and homology with coefficient in
a module. The most general comparison takes the following form.
\begin{proposition}
  Let $C$ be a small category, $M$ a right $C$\nbd-module and $N$ a
  left $C$\nbd-module. We have an isomorphism
  \[
    \HHo_{\bullet}(C,M\tens{\Z}N)\simeq \Tor{\Z C}{\bullet}(M,N).
  \]
\end{proposition}
\begin{proof}
  Let us start with the underived version of this isomorphism, that is
  to say
  \[
    (M\tens{\Z}N)\tens{\Z C^{\op}\times C}\Z\homset{C}{-}{-} \simeq M\tens{\Z C} N.
  \]
  This follows from the following abstract non-sense co-end calculus
  \[
    \begin{aligned}
     (M\tens{\Z}N)\tens{\Z C^{\op}\times C}\Z\homset{C}{-}{-} &=
                                                                \int^{(c,d)\in
                                                                \Ob(C^{\op}\times
                                                                C)}M(c)\tens{\Z}N(d)\tens{\Z}\Z\homset{C}{c}{d}\\ 
      &\simeq\int^{c \in C^{\op}}M(c)\tens{\Z}\left(\int^{d \in
        C}N(d)\tens{\Z}\Z\homset{C}{c}{d}\right)\\
                                                              &\simeq\int^{c\in C^{\op}}M(c)\tens{\Z}N(d)\\
      &=M\tens{\Z C}N,
    \end{aligned}
  \]
  where the second to last isomorphism comes from the Yoneda
  lemma. For the derived version, let $F_{\bullet} \to M$ and
  $G_{\bullet} \to N$ free resolutions. By definition, we have
  \[
    \Tot_k(P_{\bullet}\tens{\Z}Q_{\bullet})=\bigoplus_{i+j=k}P_i\tens{\Z}Q_j.
  \]
  By \cref{tensfreemod}, each $P_i\tens{\Z}Q_j$ is a free left
  $C^{\op}\times C$ module and since a sum of free objects is
  free, we get that
  $\Tot_k(P_{\bullet}\tens{\Z}Q_{\bullet})$ is free and thus
  projective by \cref{lemma:freeisproj}. It follows that
  \[
    \Tot(P_{\bullet}\tens{\Z}Q_{\bullet}) \to M\tens{\Z}N
  \]
  is a projective resolution of $M\tens{\Z}N$. Hence,
  $\HHo_{\bullet}(C,M\tens{\Z}N)$ can be taken to be the homology of
  the left hand side of the following isomorphism
  \[
    \left(\Tot(P_{\bullet}\tens{\Z}Q_{\bullet})\right)\tens{\Z C^{\op}\times C}\Z
    \homset{C}{-}{-}\simeq \Tot(P_{\bullet}\tens{\Z}Q_{\bullet}).
  \]
  The homology of the right hand side being $\Tor{\Z C}{\bullet}(M,N)$,
  this proves the result.
\end{proof}
% \begin{paragr}
%   Let $\pi \colon \Tw{C} \to C^{\op}\times C$ be the canonical
%   projection functor. As usual, we obtain an adjoint triple
%   \[
%     \begin{tikzcd}
%       \LMod{C^{\op}\times C} \ar[r,"\pi^*" description] &
%       \LMod{\Tw{C}}.\ar[l,"\pi_!"',bend right] \ar[l,"\pi_*",bend left]
%     \end{tikzcd}
%   \]
%   By a straightforward computation, one checks that, for $\M$ a
%   natural system over $C$, $\pi_!(\M)$ is given by the following formula
%   % \[
%   %   \pi_!(\B) = \bigoplus_{w}
%   % \]
% \end{paragr}
Combining the previous two propositions, we obtain the following corollary.
\begin{corollary}
  Let $C$ be a small category and $M$ a right (resp.\ left)
  $C$\nbd-module. We have
  \[
    \Ho^{BS}_{\bullet}(C,\pi^*p^*(M))\simeq \HHo_{\bullet}(C,p^*(M))\simeq \Ho_{\bullet}(C,M),
  \]
  where $p$ is the projection $C^{\op}\times C \to C^{\op}$ (resp.\
  $C^{\op}\times C \to C$).
\end{corollary}
In other words, Baues-Wirsching homology is the most general one as
it allows to recover the other ones.
\section{Homology of categories via polygraphic resolutions}
\subsection{Abelianization of $\omega$-categories}
\begin{paragr}
  Let $C$ be an \oo\nbd-category. Recall that we denote by
  $\lambda(C)$ the chain complex such that
  \[
    \lambda_n(C):=\Z C_n/\sim,
  \]
  where $\sim$ is the congruence generated by
  \[
    x \comp{i} y \sim x + y
  \]
  whenever it makes sense. The differential is defined as
  \[
    \begin{aligned}
      \lambda_{n}(C) &\to \lambda_{n+1}(C)\\
      [x] &\mapsto [\tge{x}]-[\sce{x}].
    \end{aligned}
  \]
  The fact that $\partial \circ \partial=0$ follows immediately from
  the globular identities.

  This construction is easily seen to be functorial and thus provides
  a so-called ``abelianization'' functor
  \[
    \lambda \colon \oCat \to \Chp.
  \]
  Recall that for any $i \in \N\cup \set{\omega}$, the category
  $\opCat{i}$ is a subcategory of $\oCat$, and thus it makes sense for
  an object $C$ of $\opCat{i}$ to consider the chain complex
  $\lambda(C)$.
\end{paragr}
% \begin{lemma}\label{lemma:abelpol}
%   Let $C$ be an \ook{i}\nbd-category, with $i \in \N\cup
%   \set{\omega}$, and suppose that $C$ is free on a polygraph in $\opCat{i}$. For $\Sigma_n$ a set of $n$\nbd-dimensional generators (which is not
%   unique in general if $n>i$), then the canonical map
%   \[
%     \Z \Sigma_n \to \lambda_n(C)
%   \]
%   induced by inclusion is an isomorphism.
% \end{lemma}


\begin{lemma}\label{lemma:abelpol}
  Let $C$ be an \ook{i}\nbd-category, with $i \in \N\cup\set{\omega}$
  and suppose that $C$ is free on a polygraph in $\opCat{i}$. For
  $\Sigma_n$ a set of $n$\nbd-dimensional generators of $C$, there
  exists a unique function
  \[
    \begin{aligned}
      C_n &\to \Z\Sigma_n\\
      x &\mapsto [x]
    \end{aligned}
  \]
  such that:
  \begin{itemize}
    \item if $x \in \Sigma_n$, then $[x]=x$ (seen as a generator of
      $\Z\Sigma_n$),
    \item $[x\comp{i} y]=[x]+[y]$.
    \end{itemize}
    Moreover, this function induces an isomorphism
    \[
      \lambda_n(C) \overset{\simeq}{\longrightarrow} \Z\Sigma_n.
    \]
  \end{lemma}
  \begin{paragr}
  With the same notations as the above lemma, for each $\alpha \in
  \Sigma_n$, we can also consider the projection $\Z \Sigma_n \to \Z
  \langle \alpha \rangle \simeq \Z$, and thus we get a
  function
  \[
    \wght{\alpha} \colon C_n \to \Z
  \]
  such that:
  \begin{itemize}
    \item for $\beta \in \Sigma_n$, $\wght{\alpha}(\beta)=1$ if $\beta=\alpha$ and $\wght{\alpha}(\beta)=0$
      otherwise,
    \item for $x,y \in C_n$ that are $\comp{i}$\nbd-composable, $\wght{\alpha}(x\comp{i}y)=\wght{\alpha}(x)+\wght{\alpha}(y)$.
    \end{itemize}
    It can be shown that, for each $\alpha \in \Sigma_n$, $\wght{\alpha}$ is the unique function which
    satisfies these properties. We refer to these functions as \emph{weight functions}. We thus
    have
    \[
      [x]=\sum_{\alpha \in \Sigma_n}\wght{\alpha}(x)\cdot \alpha.
    \]
\end{paragr}
The following proposition follows immediatly.
\begin{proposition}
  Let $C$ be an  \ook{i}\nbd-category, with $i \in \N\cup
  \set{\omega}$. Suppose that $C$ is free on a polygraph in
  $\opCat{i}$ and let $(\Sigma_n)_{n \geq 0}$ be a basis of $C$. The
  chain complex $\lambda_n(C)$ is isomorphic to
  \[
    \Z \Sigma_0 \leftarrow \Z \Sigma_1 \leftarrow \cdots,
  \]
  where the differential is given on generators by
  \[
    \partial(x)=[\tge{x}]-[\sce{x}].
  \]
\end{proposition}
\subsection{Polygraphic homology with coefficient}
\begin{paragr}
  Let $C$ be a small category and $u \colon c \to d$ a morphism of
  $C$. We define the category $\tr{C}{u}$ in the following fashion
  \begin{itemize}
  \item an object is a triple $(x,f,g)$, where $x$ is an object of
    $C$, $f \colon c \to x$ and $g \colon x \to d$ such that $u=gf$
    \[
      \begin{tikzcd}[column sep=tiny]
        &x\ar[dr,"g"]&\\
        c \ar[ur,"f"]\ar[rr,"u"'] & &d
      \end{tikzcd}
    \]
  \item a morphism $(x,f,g) \to (x',f',g')$ consists of a morphism $h
    \colon x \to x'$ such that $h f = f'$ and $g h=g'$,
    i.e.\ a ``morphism of factorisations''
    \[
      \begin{tikzcd}
        & x \ar[dr,"g"]\ar[dd,"h"]&\\
        c \ar[ur,"f"] \ar[dr,"f'"'] & & d \\
        & x'. \ar[ur,"g'"']& 
      \end{tikzcd}
    \]
    % such that the following diagram is commutative
    % \[
    %   \begin{tikzcd}
    %     &x'\ar[rdd,"g'"]&\\
    %     &x\ar[dr,"g"']\ar[u,"h" near start]&\\
    %     c \ar[ur,"f"']\ar[rr,"u"']\ar[uur,"f'"] & &d.
    %   \end{tikzcd}
    %   \]
    \end{itemize}
    The composition and units are defined in the obvious way. Notice that a morphism $h \colon (x,f,g) \to (x',f',g')$ is
    completely determined by the data of $f$, $h$ and $g'$, such that
    $g'hf = u$. This last equality means exactly that the pair
    $(f,g')$ is a morphism from $h$ to $u$ in the category
    $\Tw{C}$. All in all, we
    have the following useful alternative description:
    \begin{itemize}
    \item a morphism of $\tr{C}{u}$ consists of a triple $(h,f,g')$
      such that $h$ is a morphism of $C$ and $(f,g')$ is an arrow of
      $\Tw{C}$ from $h$ to $u$
      \[
        \begin{tikzcd}
          x \ar[r,"h"]& x' \ar[d,"g'"]\\
          c\ar[u,"f"] \ar[r,"u"] & d.
        \end{tikzcd}
      \]
      The source of $(h,f,g')$ is the triple $(x,f,g'h)$ and the target
      is the triple $(x,hf,g')$.
    \end{itemize}
    
    % It is useful to notice that we have an bijection of sets
    % \[
    %   \Mor(\tr{C}{u}) \simeq \coprod_{h \in \Mor(C)}\homset{\Tw{C}}{\id_{\tge{h}}}{u}.
    % \]
    % In other uords, a morphism of 
  \end{paragr}
  \begin{remark}
    Note that there are canonical projection functors
    \[
      \begin{aligned}
        \tr{C}{u} &\to \tr{C}{d}\\
        (x,f,g) &\mapsto (x,g)
      \end{aligned}
    \]
    and
    \[
      \begin{aligned}
        \tr{C}{u} &\to \cotr{C}{c}\\
        (x,f,g) &\mapsto (x,f).
      \end{aligned}
    \]
  \end{remark}
  \begin{lemma}\label{lemma:projcond}
    The canonical projection functor
    \[
      \begin{aligned}
        \tr{C}{u} &\to C\\
        (x,f,g) &\mapsto x
      \end{aligned}
    \]
    is a discrete Conduché fibration. 
  \end{lemma}
  \begin{proof}
    TODO: À écrire (mais c'est facile)
  \end{proof}
  \begin{lemma}
    The category $\tr{C}{u}$ has both an initial and terminal object.
  \end{lemma}
  \begin{proof}
    The initial object is given by $(c,\id_{c},u)$ and the terminal
    object by $(d,u,\id_{d})$.
  \end{proof}

  \begin{paragr}\label{paragr:slicepol}
    Let $C$ be a $\pair{1}{k}$\nbd-category with $k \in \set{0,1}$
    (i.e.\ either a category or a groupoid), and let
    \[
       P \to C
    \]
    be a polygraphic resolution in $\opCat{i}$ for some $i \geq
    k$. Namely, this means that $P$ is free on a \ook{i}-polygraph and
    $P\to C$ is a trivial fibration. For $x$ an $n$\nbd-cell of $P$,
    let us denote by $\overline{x}$ its image in $C$ by this trivial
    fibration. Note that for $n>1$, this is necessarily a degenerate
    $1$\nbd-cell because $C$ is a $1$\nbd-category. We shall abusively
    use the
    same notation $\overline{x}$ for this unique $1$\nbd-cell of $C$.
  \end{paragr}
  \begin{paragr}
    For every arrow $u\colon c \to d$ of $C$, we define an
    \ook{i}\nbd-category $\tr{P}{u}$ as the following fibred product
    \[
      \begin{tikzcd}
        \tr{P}{u} \ar[r] \ar[d] & P \ar[d] \\
        \tr{C}{u} \ar[r] & C.
        \ar[from=1-1,to=2-2,phantom,"\lrcorner" very near start]
      \end{tikzcd}
    \]
    Explicitly, an $n$\nbd-cell of $\tr{P}{u}$ consists of a
      triple $(\alpha,f,g)$, where $\alpha$ is an $n$\nbd-cell of $P$,
      $f \colon c \to \src{0}\overline{\alpha}$ and $g \colon
      \tgt{0}\overline{\alpha}\to d$ are arrows of $C$ such that
      $g\overline{\alpha}f=u$ (with the convention that for $n=0$,
      $\src{0}\alpha=\tgt{0}\alpha=\alpha$). The sources
      and targets are given by
      \[
        \sce{(\alpha,f,g)}=\begin{cases}
          (\sce{}\alpha,f,g)& \text{ if } n>1, \\
          (\sce{}\alpha,f,g\overline{\alpha})& \text{ if } n=1
          \end{cases}
        \]
        and
        \[
          \tge{(\alpha,f,g)}=\begin{cases}
          (\tge{}\alpha,f,g)& \text{ if } n>1, \\
          (\tge{}\alpha,\overline{\alpha}f,g)& \text{ if } n=1.
          \end{cases}
        \]
    By stability of the class of discrete Conduché fibrations under
    pullback, it follows from \cref{lemma:projcond} that the morphism
    $\tr{P}{u}\to P$ is a discrete Conduché fibration, and thus by
    \ref{} $\tr{P}{u}$ is free on an
    \ook{i}\nbd-polygraph. Explicitly, if $\Sigma_n$ is an
    $n$\nbd-base of $P$, then an $n$\nbd-base of $\tr{P}{u}$ is given
    by
    \[
      \tr{\Sigma_n}{u}:=\setof{(\alpha,f,g)}{\alpha \in \Sigma_n}.
    \]
    In fact,
    we will show a stronger freeness property in
    \cref{prop:cofprojresol} below. Similarly, by
    stability of the class of trivial fibration by pullback, it
    follows that \[\tr{P}{u} \to \tr{C}{u}\] is a trivial fibration,
    and thus a polygraphic resolution in $\opCat{i}$.
  \end{paragr}
  \begin{definition}
    A \oo\nbd-category $C$ is \emph{oplax left contractible} if
    there exists an object $x_0$ of $C$ and an oplax transformation
    \[
      \begin{tikzcd}
        C \ar[r] \ar[dr,""{name=toto},"\id_{C}"'] & \Term
        \ar[d," x_0"] \\
        & C,
        \ar[from=toto,to=1-2,Rightarrow,"\eta"]
      \end{tikzcd}
    \]
    where $\Term$ is the terminal category.% , such that the composite
    % \[
    %   \begin{tikzcd}
    %     \Term \ar[r,"x_0"]& C\ar[r,bend left,""{name=A,below}]
    %     \ar[r,bend right,""{name=B,above}]& C
    %     \ar[from=A,to=B,"\eta",Rightarrow]
    %   \end{tikzcd}
    % \]
    % is the identity transformation on $\Term
    % \overset{x_0}{\longrightarrow} C$.

    Dually, $C$ is \emph{oplax right contractible} if there exists an
    object $y_0$ of $C$ and an oplax transformation
          \[
      \begin{tikzcd}
        C \ar[r] \ar[dr,""{name=toto},"\id_{C}"'] & \Term
        \ar[d," y_0"] \\
        & C.
        \ar[from=1-2,to=toto,Rightarrow,"\epsilon"]
      \end{tikzcd}
    \]
    %  such that the composite
    % \[
    %   \begin{tikzcd}
    %     \Term \ar[r,"y_0"]& C\ar[r,bend left,""{name=A,below}]
    %     \ar[r,bend right,""{name=B,above}]& C
    %     \ar[from=A,to=B,"\epsilon",Rightarrow]
    %   \end{tikzcd}
    % \]
    % is the identity transformation on $\Term
    % \overset{y_0}{\longrightarrow} C$.
  \end{definition}
  \begin{example}
    Since oplax transformations between functors into a
    ($1$\nbd-)category are simply natural transformations, it follows
    that a ($1$\nbd-)category that has an terminal (resp.\ initial) object is oplax
    left (resp.\ right) contractible.
  \end{example}
  % \begin{remark}
  %   In the case that $C$ is a $1$\nbd-category, this is equivalent to
  %   the usual notions of initial and terminal objects, since oplax
  %   transformations then reduces to natural transformations.
  % \end{remark}
  \begin{lemma}\label{lemma:oplaxterm}
  The \oo\nbd-category $\tr{P}{u}$ is both left and right oplax contractible.
  \end{lemma}
  \begin{proof}
    As already observed, $\tr{C}{u}$ both has a terminal and a initial
    objects, hence it is both oplax left and right contractible. Let us treat the case of
    oplax left contractibility of $\tr{P}{u}$, as the other one can be
    deduced by duality.

    Since $P \to C$ is a trivial fibration, there exists an object $x$
    of $P$ which is sent to $d=\tge{u}$. It follows that the object
    $(x,u,\id_d)$ of $\tr{P}{u}$ is sent to the terminal object
    $(d,u,\id_d)$ of $\tr{C}{u}$. Consider now the following
    commutative diagram
    \[
      \begin{tikzcd}
        \Sph{0}\tens{\omega,i}\tr{P}{u} \ar[r] \ar[d] & \tr{P}{u} \ar[d]\\
        \Dsk{1}\tens{\omega,i}\tr{P}{u} \ar[r] & \tr{C}{u},
      \end{tikzcd}
    \]
    where the top arrow is
    \[
      \begin{tikzcd}[column sep=huge]
      \Sph{0}\tens{\omega,i}\tr{P}{u} \simeq \tr{P}{u}\coprod\tr{P}{u}
      \ar[r,"{\langle\id,\mathrm{cst}_{(x,u,\id_d)}\rangle}"]&\tr{P}{u}
      \end{tikzcd}
    \]
    and the bottow arrow is
    \[
      \Dsk{1}\tens{\omega,i}\tr{P}{u} \to
      \Dsk{1}\tens{\omega,i}\tr{C}{u} \overset{\eta}{\longrightarrow}
       \tr{C}{u},
     \]
     where $\eta$ is the oplax transformation witnessing $(d,u,\id_d)$
     as a terminal object of $\tr{C}{u}$. 
    Since $\tr{P}{u}\to \tr{C}{u}$ is a trivial fibration and
    $\Sph{0}\tens{\omega,i}\tr{P}{u} \to \Dsk{1}\tens{\omega,i}\tr{P}{u}$ is a
    cofibration, there exists a lift $\Dsk{1}\tens{\omega,i}\tr{P}{u}
    \to \tr{P}{u}$ that makes the previous square commutative. This
    gives the desired oplax transformation.
    % $\tr{P}{w} \to \tr{C}{w}$ is a trivial fibration, it is in
    % particular surjective on objects. Hence, there exists an object
    % $x$ of $P$
  \end{proof}
  \begin{paragr}
      Keeping the notation of \ref{paragr:slicepol}, the construction of $u\mapsto \tr{C}{u}$ extends to a canonical functor
    \[
        \tr{C}{-}\colon \Tw{C} \to \Cat,
      \]
      where an arrow $(f,g) \colon u \to u'$ of $\Tw{C}$ is sent to the functor
      \[
        \begin{aligned}
          \tr{C}{u} &\to \tr{C}{u'}\\
          (x,p,q) &\mapsto (x,pf,gq)\\
          (h,p,q) &\mapsto (h,pf,gq).
        \end{aligned}
      \]
      
    A similar construction allows to define a functor
    \[
      \begin{aligned}
       \tr{P}{-}\colon \Tw{C} &\to \opCat{i} \\
        u &\mapsto \tr{P}{u}.
      \end{aligned}
    \]
  \end{paragr}
  \begin{proposition}\label{prop:cofprojresol}
    Let $C$ be a $\pair{1}{k}$\nbd-category with $k \in \set{0,1}$
    (i.e.\ either a category or a groupoid), and $P \to C$ a
    polygraphic resolution in $\opCat{i}$, for some $i\in
    \N\cup\set{\omega}$ such that $i\geq k$. The functor
    \[
      \tr{P}{-} \colon \Tw{C} \to \opCat{i},
    \]
   is cofibrant for the projective model structure on $\fun{\Tw{C}}{\opCat{i}}$.
  \end{proposition}
  \begin{proof}
    Let $u\colon c \to d$ be a morphism of $C$ and let $(\Sigma_n)_n$
    be a basis of $P$. As we have already seen, the canonical morphism
    \[
      \tr{P}{u} \to P,
    \]
    is a Conduché fibration and thus $\tr{P}{u}$ is free on an
    \ook{i}\nbd-polygraph, with a basis given by $(\tr{\Sigma_n}{u})_n$
    with
    \[
      \tr{\Sigma_n}{u}:=\setof{(\alpha,f,g)}{\alpha \in
        \Sigma_n}\simeq \coprod_{\alpha \in \Sigma_n}\homset{\Tw{C}}{\overline{\alpha}}{u}.
    \]
    In other words, for any $n > 0$, we have the following pushout
    square in \ook{i}
    \[
      \begin{tikzcd}
        \displaystyle\coprod_{\alpha \in \Sigma_n}\coprod_{\homset{\Tw{C}}{\overline{\alpha}}{u}}\Sph{n-1} \ar[r]\ar[d] &
        \trk{n-1}(\tr{P}{u}) \ar[d] \\
            \displaystyle \coprod_{\alpha \in \Sigma_n}\coprod_{\homset{\Tw{C}}{\overline{\alpha}}{u}}\Dsk{n} \ar[r] &
            \trk{n}(\tr{P}{u})
            \ar[from=1-1,to=2-2, phantom, very near end,"\ulcorner"]
      \end{tikzcd}
    \]
    It is straightforward to check that this pushout square is
    natural in $u$, and thus defines a pushout square in the category
    $\fun{\Tw{C}}{\opCat{i}}$. Since for any morphism $v$ of $C$, the morphism
    \[
      \coprod_{\homset{\Tw{C}}{v}{-}}\Sph{n-1}\to \coprod_{\homset{\Tw{C}}{v}{-}}\Dsk{n}
    \]
    is a cofibration for the projective model structure on
    $\fun{\Tw{C}}{\opCat{i}}$ induced by the folk model structure on
    $\opCat{i}$, it follows that
    \[
      \trk{n-1}(\tr{P}{-}) \to \trk{n}(\tr{P}{-})
    \]
    is also a cofibration for the projective model structure. By
    stability of cofibrations by transfinite composition, it follows
    that $\tr{P}{-}$ is cofibrant.
  \end{proof}
  \begin{paragr}
    Keeping the notations from the previous proposition, by applying
    the abelianization functor $\lambda$, we get a augmented chain
    complex of left $\Tw{C}$\nbd-modules
    \[
      \cst{\Z} \leftarrow \lambda_0(\tr{P}{-}) \leftarrow
      \lambda_1(\tr{P}{-}) \leftarrow \cdots,
    \]
    where the augmentation $\cst{\Z} \leftarrow \lambda_0(\tr{P}{-})$
    is induced by the canonical morphism $\tr{P}{u} \to \Term$, which
    is obviously natural in $u$.
  \end{paragr}
  The fundamental theorem of polygraphic homology is the following.
  \begin{theorem}\label{fundthm}
       Let $C$ be a $\pair{1}{k}$\nbd-category with $k \in \set{0,1}$
    (i.e.\ either a category or a groupoid), and $P \to C$ a
    polygraphic resolution in $\opCat{i}$, for some $i\in
    \N\cup\set{\omega}$ such that $i\geq k$. The augmented chain
    complex of left $\Tw{C}$\nbd-modules
        \[
      \cst{\Z} \leftarrow \lambda_0(\tr{P}{-}) \leftarrow
      \lambda_1(\tr{P}{-}) \leftarrow \cdots,
    \]
    is a projective resolution of $\cst{\Z}$.
  \end{theorem}
  \begin{proof}
    Recall that the abelianization functor $\lambda \colon \opCat{i}
    \to \Chp$ is monoidal with respect to the Gray tensor product on
    $\opCat{i}$ and the usual tensor product of chain complexes on
    $\Chp$, and is left Quillen with respect to the folk model
    structure on $\opCat{i}$ and the \emph{projective} model structure
    on $\Chp$. The fact that each $\lambda_i(\tr{P}{-})$ is projective then
    follows from \cref{prop:cofprojresol} and the fact that $\lambda$
    is left Quillen.
    The fact that the augmented chain complex $\Z \leftarrow
    \lambda_{\bullet}(\tr{P}{u})$ is a resolution of $\Z$ for any $u$
    follows from \cref{lemma:oplaxterm} and the fact that, since $\lambda$ is
    monoidal, it sends oplax transformation to homotopies of chain complexes.  This proves the result since homology in the category of
    left $\Tw{C}$\nbd-modules is computed pointwise.
  \end{proof}
    \lgcomment{Je pense que c'est là que se cache le loup dans la
    section 23.3 du polybook. Ce complexe
  de chaînes est décrit ``à la main'' et on retrouve bien les
  $n$-chaines du paragrphe
  \ref{paragr:descres} ci-contre. Par contre, il me semble que la
  description de la différentielle donnée dans le polybook n'est pas
  suffisante car ne sont décrits que les images des générateurs de la
  forme $(\alpha,\id,\id)$. Dans l'approche que j'ai pris je sais formellement que
  cette différentielle existe car
  je sais que complexe de chaînes est l'abélianisé d'un polygraphe
  (grâce à la technique avec Conduché), et la description de la
  différentielle se déduit alors des formules générales.}
  \begin{paragr}\label{paragr:descres}
    We keep the notations from \cref{fundthm}. We can be more
    precise as the resolution
    \[
      \cst{\Z} \leftarrow \lambda_0(\tr{P}{-}) \leftarrow
      \lambda_1(\tr{P}{-}) \leftarrow \cdots
    \]
    is actually a \emph{free} resolution of left
    $\Tw{C}$\nbd-modules. %  To see
    % that, let $(\Sigma_n)_{n}$ be a basis of $P$. It follows from
    % \ref{} that for every morphism $w\colon c \to d$ of $C$, $\tr{P}{w}$ as for
    % basis $(\tr{\Sigma_n}{u})_n$, where
    % \[
    %   \tr{\Sigma_n}{w}:=\coprod_{x \in \Sigma_n}\homset{\Tw{C}}{\pi(\tgt{0}(x))}{w},
    % \]
    % that is to say that a generating $n$\nbd-cell of $\tr{P}{w}$ consists of a
    % triple $(x,f,g)$, where $x \in \Sigma_n$, $f \colon c \to
    % \pi(\tgt{0}(x))$ and $g \colon \pi(\tgt{0}(x)) \to d$ are arrows
    % of $C$ such that
    % \[
    %   w=gf.
    % \]
    Indeed, if $(\Sigma_n)_n$ is a basis of $P$, it follows from
    \cref{lemma:abelpol} and paragraph \ref{paragr:slicepol} that the resolution takes the form
    \[
       \cst{\Z} \leftarrow \bigoplus_{\alpha \in \Sigma_0}\Z
       \homset{\Tw{C}}{\id_{\overline{\alpha}}}{-} \leftarrow \bigoplus_{\alpha \in \Sigma_1}\Z
       \homset{\Tw{C}}{\overline{\alpha}}{-} \leftarrow % \bigoplus_{\alpha \in \Sigma_2}\Z
       % \homset{\Tw{C}}{\overline{\alpha}}{-} \leftarrow
       \cdots
     \]
     Let $u$ be an arrow of $C$. For $n>1$, the differential
     \[
       \partial \colon \bigoplus_{\alpha \in
         \Sigma_{n}}\Z\homset{\Tw{C}}{\overline{\alpha}}{u} \to \bigoplus_{\alpha \in
         \Sigma_{n-1}}\Z\homset{\Tw{C}}{\overline{\alpha}}{u}
     \]
     is given on generators by the formula
     \[
       \partial(\alpha,f,g)=\sum_{\beta \in
         \Sigma_{n-1}}\sum_{(h,i)\in
         \homset{\Tw{C}}{\overline{\beta}}{u}}(\wght{(\beta,h,i)}(\tgt{\alpha},f,g)
       -\wght{(\beta,h,i)}(\src{\alpha},f,g))\cdot(\beta,h,i),
     \]
     and for $n=1$, by the formula
     \[
       \partial(\alpha,f,g)=\sum_{\beta \in \Sigma_0}\sum_{(h,i)\in \homset{\Tw{C}}{\id_{\overline{\beta}}}{u}}\wght{(\beta,h,i)}(\tgt{\alpha},\overline{\alpha}f,g)-\wght{(\beta,h,i)}(\src{\alpha},f,g\overline{\alpha}).
     \]
  \end{paragr}
  We then deduce the following theorem, which says that the homology
  of $C$ with coefficient in a natural system is given by the
  polygraphic complex ``twisted'' by the natural system.
  \begin{theorem}\label{polhom}
    Let $C$ be a $\pair{1}{k}$\nbd-category with $k \in \set{0,1}$
    (i.e.\ either a category or a groupoid) and $\M$ be a natural
    system over $C$. Then, for any $i\in
    \N\cup\set{\omega}$ such that $i\geq k$, and any $P \to C$ 
    polygraphic resolution in $\opCat{i}$, with a given basis
    $(\Sigma_n)_{n\geq 0}$, the Baues-Wirsching
    homology $\Ho^{BS}_{\bullet}(C,\M)$ is isomorphic to the homology
    of the chain complex
\[         \bigoplus_{\alpha \in \Sigma_0}\M_{\id_\alpha}\leftarrow \bigoplus_{\alpha \in
        \Sigma_1}\M_{\overline{\alpha}} \leftarrow \bigoplus_{\alpha \in
        \Sigma_2}\M_{\overline{\alpha}} \leftarrow \cdots,
    \]
    where the differential is given by
    \[
      \partial(\alpha,m)=\sum_{\beta \in
        \Sigma_{n-1}}\left(\beta,\sum_{(f,g)\in
        \homset{\Tw{C}}{\beta}{\overline{\alpha}}}(\wght{(\beta,f,g)}(\tgt{\alpha})-\wght{(\beta,f,g)}(\src{\alpha}))\cdot
      (f^*,g_*)(m)\right)
  \]
  \end{theorem}

  % \begin{corollary}\label{polhom}
  %   Let $C$ be a $\pair{1}{k}$\nbd-category with $k \in \set{0,1}$
  %   (i.e.\ either a category or a groupoid) and $\M$ be a natural
  %   system over $C$. Then, for any $i\in
  %   \N\cup\set{\omega}$ such that $i\geq k$, and any $P \to C$ 
  %   polygraphic resolution in $\opCat{i}$, the Baues-Wirsching
  %   homology $\Ho^{BS}_{\bullet}(C,\M)$ is isomorphic to the homology
  %   of the chain complex
  %   \[
  %      \M \tens{\Z \Tw{C}}\lambda_0(\tr{P}{-})\leftarrow
  %     \M \tens{\Z \Tw{C}}\lambda_1(\tr{P}{-}) \leftarrow \cdots
  %   \]
  % \end{corollary}

  % \begin{paragr}
  %   Keeping the notations from \cref{polhom}, we can be more
  %   precise. Let $(\Sigma_n)_n$ be a basis of $P$, then the
  %   Baues-Wirsching homology
  %   of $C$ with coefficient in $\M$ is isomorphic to the chain complex
  %   \[
  %     \bigoplus_{x \in \Sigma_0}\M_{\id_x}\leftarrow \bigoplus_{x \in
  %       \Sigma_1}\M_{\overline{x}} \leftarrow \bigoplus_{x \in
  %       \Sigma_2}\M_{\overline{x}} \leftarrow \cdots
  %   \]
  %   In particular, if we take $\M=\cst{\Z}$, then we recover the usual
  %   complex defining the polygraphic homology. This shows in particular that
  %   the homology of a $1$\nbd-category (resp.\ a groupoid) can be
  %   computed using polygraphic resolutions in strict
  %   \ook{1}\nbd-categories (resp.\ strict \oo\nbd-groupoids). The same
  %   goes for any kind of coefficient (e.g.\ $C$\nbd-module).
  % \end{paragr}
% \begin{paragr}
%   Let $C$ be a $(1,k)$\nbd-category for $k\in \set{0,1}$ (i.e.\ either
%   a groupoid or a category), let
%   \[
%     \pi \colon \frcat{P} \to C
%   \]
%   be a polygraphic resolution in $\opCat{i}$ with $i\in
%   \N\cup\set{\omega}$ such that $i \geq k$. Notice that since $C$ is a
%   $1$\nbd-category, for any $n$\nbd-cell of $P$ with $n\geq 1$,
%   $\pi(x)$ is the iterated identity of a unique $1$\nbd-cell (i.e.\ an
%   arrow) of $C$. We will also denote this arrow by $\pi(x)$.

  
%   Now, let $\M$ a natural system over $C$, we define a chain complex
%   $\lambda(P,\M)$ as follows. For $n=0$, we set
%   \[
%     \lambda_0(P,\M):=\bigoplus_{x \in P_0}\M_{\id_x}
%   \]
%   For $n\geq 1$, we set
%   \[
%     \lambda_n(P,\M):=\bigoplus_{x \in P_n} \M_{\pi(x)}.
%   \]
%   For $n=1$, the differential $\partial_1$ is defined as
%   \[
%     \begin{aligned}
%       \lambda_1(P,\M)&\to \lambda_0(P,\M)\\
%       (x,m)&\mapsto (t(x),)-(s(x),)
%     \end{aligned}
%   \]
% \end{paragr}
\end{document}