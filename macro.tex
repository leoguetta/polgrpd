
%ENVIRONMENTS

\swapnumbers
\theoremstyle{plain}
\newtheorem{theorem}{Theorem}[subsubsection]
\newtheorem{corollary}[subsubsection]{Corollary}
\newtheorem{proposition}[subsubsection]{Proposition}
\newtheorem{lemma}[subsubsection]{Lemma}

\theoremstyle{definition}
\newtheorem{definition}[subsubsection]{Definition}
\newtheorem{example}[subsubsection]{Example}
\newtheorem{paragr}[subsubsection]{} %Environnement "paragraphe"

\theoremstyle{remark}
\newtheorem{remark}[subsubsection]{Remark}

% Cleveref definitions
\crefname{theorem}{Theorem}{Theorems}
\crefname{corollary}{Corollary}{Corollaries}
\crefname{proposition}{Proposition}{Propositions}
\crefname{lemma}{Lemma}{Lemmas}

\crefname{definition}{Definition}{Definitions}
\crefname{example}{Example}{Examples}

\crefname{rem}{Remark}{Remarks}
\crefname{paragr}{}{}

% MATH G

\newcommand{\nmbr}[1]{\mathbb #1}
\newcommand{\N}{\nmbr{N}}
\newcommand{\Z}{\nmbr{Z}}
\newcommand{\Q}{\nmbr{Q}}
\newcommand{\R}{\nmbr{R}}
\newcommand{\C}{\nmbr{C}}

\newcommand{\set}[1]{\left\{#1\right\}} 
\newcommand{\setof}[2]{\left\{ #1\; | \; #2\right\}}
\newcommand{\pair}[2]{(#1,#2)}       %pair

% CATEGORIES

% general

\newcommand{\psh}[1]{\widehat{#1}} % presheaves
\newcommand{\plim}{\varprojlim} % limit
\newcommand{\ilim}{\varinjlim} % colimit
\newcommand{\source}{\mathsf{s}} % source
\newcommand{\target}{\mathsf{t}} % target
\newcommand{\cosource}{\sigma} % source
\newcommand{\cotarget}{\tau} % target
\newcommand{\geninc}{\mathsf{i}} % inclusion of generators
\newcommand{\cmp}{\circ} % composition
\newcommand{\unit}[1]{1^{#1}} % unit
\newcommand{\homset}[3]{#1(#2,#3)} % homset in #1 from #2 to #3
\newcommand{\fun}[2]{\mathbf{Hom}(#1,#2)} %Functor categories
\newcommand{\Ob}{\mathrm{Ob}} %Set of objects
\newcommand{\prd}[1]{\underset{#1}{\times}} 
\newcommand{\sumalg}[1]{\underset{#1}{+}} 
\newcommand{\pbck}[3]{#1\prd{#3} #2} % pullback #1 x_#3 #2
\newcommand{\psht}[3]{#1\sumalg{#3} #2} % pushout #1 +_#3 #2
\newcommand{\op}{{\mathrm{op}}} % opposite category symbol
\newcommand{\el}[1]{\mathrm{el}(#1)} %Category of elements of a presheaf
\newcommand{\tr}[2]{#1/#2} %Slice categories

% special categories

\newcommand{\ctg}[1]{\mathbf{#1}}  % large category
\newcommand{\Set}{\ctg{Set}}    % category of sets
\newcommand{\Cat}{\ctg{Cat}}   % category of small categories
\newcommand{\nCat}[1]{\Cat_{#1}}  % n-categories
\newcommand{\npCat}[2]{\Cat_{#1,#2}}  % (n,p)-categories
\newcommand{\opCat}[1]{\Cat_{\omega,#1}} % (omega,p)-categories
\newcommand{\nCatp}[1]{\Cat_{#1}^{+}} % cellular extensions
\newcommand{\npCatp}[2]{\Cat_{#1,#2}^{+}} % cellular extensions
\newcommand{\oCat}{\nCat{\omega}} % omega-categories
\newcommand{\glob}{\ctg{G}}% site of globular sets
\newcommand{\Glob}{\ctg{Glob}} 
\newcommand{\nGlob}[1]{\Glob_{#1}} % n-globular sets
\newcommand{\oGlob}{\nGlob{\omega}} % globular sets
\newcommand{\Pol}{\ctg{Pol}} 
\newcommand{\nPol}[1]{\Pol_{#1}} % n-polygraphs
\newcommand{\oPol}{\nPol{\omega}} % polygraphs
\newcommand{\npPol}[2]{\Pol_{#1,#2}} % (#1,#2)-polygraphs
\newcommand{\opPol}[1]{\npPol{\omega,#1}} % (omega,#1)-polygraphs
\newcommand{\Grpd}{\ctg{Grpd}}
\newcommand{\nGrpd}[1]{\Grpd_{#1}} % n-groupoids
\newcommand{\oGrpd}{\nGrpd{\omega}} %omega-groupoids
\newcommand{\Ch}[1]{\ctg{Ch}_{#1}} % chain complexes
\newcommand{\Chp}{\Ch{\geq 0}} % positive chaincomplexes
\newcommand{\Ab}{\ctg{Ab}} % abelian groups
\newcommand{\Term}{\ctg{1}} % terminal category
\newcommand{\Grph}{\ctg{Grph}}  % graphs
\newcommand{\Mon}{\ctg{Mon}} %  monoids
\newcommand{\Spl}{\Delta} % simplicial category
\newcommand{\SSet}{\psh{\Spl}} % simplicial sets

\newcommand{\RMod}[1]{\mathrm{Mod}_{#1}} %category of right modules
\newcommand{\LMod}[1]{{}_{#1}\mathrm{Mod}} %category of left modules (un peu dégueu...)

% FUNCTORS
\newcommand{\functor}[1]{\mathsf{#1}} % style for functor names
\newcommand{\trk}[1]{\functor{tr}_{#1}}%truncation functor to dim #1
\newcommand{\inc}[1]{\functor{inc}_{#1}}%inclusion functor from dim #1
\newcommand{\fgf}[1]{\functor{U}_{#1}} %forgetful functor
\newcommand{\frf}[1]{\functor{F}_{#1}} %free functor
% ARROWS

\newcommand{\doubl}[2]{\ar@<2pt>[l]^{#2}\ar@<-2pt>[l]_{#1}}
\newcommand{\dbl}{\doubl{}{}}
\newcommand{\doubr}[2]{\ar@<2pt>[r]^{#2}\ar@<-2pt>[r]_{#1}}
\newcommand{\dbr}{\doubr{}{}}
\newcommand{\doubld}[2]{\ar@<2pt>[ld]^{#2}\ar@<-2pt>[ld]_{#1}}
\newcommand{\dbld}{\doubld{}{}}


% OMEGA-CAT BASICS

\newcommand{\comp}[1]{\ast_{#1}} % composition along dim #1
\newcommand{\sce}[1]{\source_{#1}} % source of dim #1
\newcommand{\tge}[1]{\target_{#1}} % target of dim #1
\newcommand{\cosce}[1]{\cosource_{#1}} % cosource from dim #1
\newcommand{\cotge}[1]{\cotarget_{#1}} % cotarget from dim #1
\newcommand{\frcat}[1]{#1^*} % free category generated by a polygraph
\newcommand{\frgp}[1]{#1^\top} % free groupoid generated by a polygraph

 \newcommand{\oo}{$\omega$}
 \newcommand{\ook}[1]{$\pair{\omega}{#1}$}
\newcommand{\clx}[2]{\pair{#1}{#2}}% cellular extension of #1 by #2
\newcommand{\gni}[1]{\geninc_{#1}}% inclusion of #1-generators

\newcommand{\Sph}[1]{\mathbb{S}^{#1}} %n-sphere
\newcommand{\Dsk}[1]{\mathbb{D}_{#1}} %n-disk
\newcommand{\trun}[1]{\tau_{#1}} %truncation functor

% FORMAL EXPRESSIONS

\newcommand{\expr}[1]{\mathcal{E}[#1]} %formal expressions on #1
\newcommand{\wtx}[1]{\mathcal{W}[#1]} % well typed expressions on #1
\newcommand{\wtxeq}[1]{\fcls{\wtx{#1}}} % equiv classes of well typed expressions on #1
\newcommand{\formal}[1]{\mathsf{#1}} % formal expressions style
\newcommand{\fcst}[1]{\formal{c}_{#1}} % constant
\newcommand{\finv}[1]{\formal{c}^{-}_{#1}}% formal inverse
\newcommand{\fid}[1]{\formal{i}_{#1}} % formal identity
\newcommand{\fcomp}[1]{\bm{\star}_{#1}} % formal composition
\newcommand{\type}[3]{#1:#2\to #3} %typing rule
\newcommand{\frel}{\sim} % relation on formal expressions
\newcommand{\freq}{\simeq} %equivalence among formal expressions
\newcommand{\cans}[1]{\rho^{#1}} % canonical surjection
\newcommand{\fcls}[1]{\widetilde{#1}}% equivalence class of  #1
\newcommand{\finc}[1]{\formal{c}^{#1}}% inclusion of #1 in W[#1]
                              
% HOMOLOGY AND HOMOLOGICAL ALGEBRA

 \newcommand{\tens}[1]{\otimes_{#1}} %tensor product
\newcommand{\Tor}[2]{\functor{Tor}^{#1}_{#2}} %Tor
\newcommand{\cst}[1]{\underline{#1}} %Constant module
\newcommand{\Ho}{\functor{H}} %Homology


%OTHERS

\newcommand{\nbd}{\nobreakdash} %Shortcut
\newcommand{\extend}[1]{\overline{#1}} % extension of #1

%COMMENTS

\newcommand{\lgcomment}[1]{
    \marginnote{\tiny\textcolor{red}{LG: #1}}}

\newcommand{\fmcomment}[1]{
    \marginnote{\tiny\textcolor{red}{FM: #1}}}